% !TEX TS-program = xelatex
% !TEX encoding = UTF-8 Unicode
% -*- coding: UTF-8; -*-
% vim: set fenc=utf-8

%%%%%%%%%%%%%%%%%%%%%%%%%%%%%%%%%%%%%%%%%%%%%%%%%%%%%%%%%%%%%%%%%
%% CV.tex
%% <https://github.com/zachscrivena/simple-resume-cv>
%% This is free and unencumbered software released into the
%% public domain; see <http://unlicense.org> for details.
%%%%%%%%%%%%%%%%%%%%%%%%%%%%%%%%%%%%%%%%%%%%%%%%%%%%%%%%%%%%%%%%%

% See "README.md" for instructions on compiling this document.

\documentclass[letterpaper,MMMyyyy,nonstopmode]{simpleresumecv}
% Class options:
% a4paper, letterpaper, nonstopmode, draftmode
% MMMyyyy, ddMMMyyyy, MMMMyyyy, ddMMMMyyyy, yyyyMMdd, yyyyMM, yyyy

%%%%%%%%%%%%%%%%%%%%%%%%%%%%%%%%%%%%%%%%%%%%%%%%%%%%%%%%%%%%%%%%%
%% PREAMBLE.
%%%%%%%%%%%%%%%%%%%%%%%%%%%%%%%%%%%%%%%%%%%%%%%%%%%%%%%%%%%%%%%%%

% CV Info (to be customized).
\newcommand{\CVAuthor}{Chang Min Park}
\newcommand{\CVTitle}{Chang Min Park's CV}
\newcommand{\CVNote}{CV compiled on {\today}}
\newcommand{\CVWebpage}{https://changminpark.github.io/}

% PDF settings and properties.
\hypersetup{
pdftitle={\CVTitle},
pdfauthor={\CVAuthor},
pdfsubject={\CVWebpage},
pdfcreator={XeLaTeX},
pdfproducer={},
pdfkeywords={},
unicode=true,
bookmarks=true,
bookmarksopen=true,
pdfstartview=FitH,
pdfpagelayout=OneColumn,
pdfpagemode=UseOutlines,
hidelinks,
breaklinks}

% Shorthand.
\newcommand{\Code}[1]{\mbox{\textbf{#1}}}
\newcommand{\CodeCommand}[1]{\mbox{\textbf{\textbackslash{#1}}}}

%%%%%%%%%%%%%%%%%%%%%%%%%%%%%%%%%%%%%%%%%%%%%%%%%%%%%%%%%%%%%%%%%
%% ACTUAL DOCUMENT.
%%%%%%%%%%%%%%%%%%%%%%%%%%%%%%%%%%%%%%%%%%%%%%%%%%%%%%%%%%%%%%%%%

\begin{document}

%%%%%%%%%%%%%%%
% TITLE BLOCK %
%%%%%%%%%%%%%%%

\Title{\CVAuthor}

\begin{SubTitle}
University at Buffalo, The State University of New York
\par
\href{mailto:cpark22@buffalo.edu}
{cpark22@buffalo.edu}
\,\SubBulletSymbol\,
+1\,(716)\,598-7331
\,\SubBulletSymbol\,
\href{\CVWebpage}
{\url{\CVWebpage}}
\end{SubTitle}

\begin{Body}

%%%%%%%%%%%%%%%
%% Interests %%
%%%%%%%%%%%%%%%
\vspace{-2ex}
\Section
{Interests}
{Interests}
{PDF:Interests}

\Entry
Secure Image Display using Trusted Execution Environment (TEE), Systems Challenges in Mobile Systems, 
Automated Software Analysis, and UI Testing.

%%%%%%%%%%%%%%%%%%%%%%
%% TECHNICAL SKILLS %%
%%%%%%%%%%%%%%%%%%%%%%
\Section
{Technical Skills}
{Technical Skills}
{PDF:Technical Skills}

\Entry
ARM TrustZone, Android Internals and App Development, Bytecode Instrumentation Tools (Soot),
Firebase Realtime Database, AWS EC2/RDS, Java, Python, C, and Linux OS

%%%%%%%%%%%%%%%
%% EDUCATION %%
%%%%%%%%%%%%%%%
\Section
{Education}
{Education}
{PDF:Education}

\Entry
\textbf{University at Buffalo}, The State University of New York

\Gap
\BulletItem
Ph.D. in Computer Science and Engineering
\hfill
\DatestampYD{Aug}{ `17} --
\DatestampY{Present}
\begin{Detail}
\SubBulletItem
Advisor:
Prof.~Steven~Y.~Ko, 
Co-advisor:
Prof.~Karthik~Dantu
\SubBulletItem
Focus:
Systems Challenges in Mobile Computing
\end{Detail}

\Gap
\BulletItem
B.S. in Computer Science
\hfill
\DatestampYD{Aug}{ `11} --
\DatestampYD{May}{ `17}
\begin{Detail}
\SubBulletItem
Magna Cum Laude
\SubBulletItem
Summer `12: Study Abroad Program at Yonsei University in Republic of Korea
\SubBulletItem
Jun `13 -- Mar `15: Served Military Service in Republic of Korea
\end{Detail}

\BigGap
\Entry
\textbf{Relevant Courses:} Advanced Computer Systems, Advanced Programming Languages, 
Operating Systems, Distributed Systems, Realtime Embedded Systems, Modern Network Concepts, 
Data Intensive Computing, Computer Security, Algorithms Analysis \& Design,
VLSI Electronics, and Theory of Computation.


%%%%%%%%%%%%%%%%%%%%%%%
%% RESEARCH OVERVIEW %%
%%%%%%%%%%%%%%%%%%%%%%%
\Section
{Research Overview}
{Research Overview}
{PDF:ResearchOverview}

\Entry
Through my undergraduate and graduate studies, my research has focused on mobile systems.

\Gap
\BulletItem
\textbf{Rushmore} [MobiSys `21] is a system that securely displays static or animated images 
using ARM TrustZone. The core functionality of Rushmore is to securely decrypt and display encrypted 
images from a trusted party on a mobile device. By leveraging IPU's multiple display channels and
a fast stream cipher called ChaCha20, the system provides frame rates around or higher than 30 FPS 
for displaying encrypted animated images. Rushmore also enables a novel application of visual 
cryptogrpahy.

\hspace{2ex}
\textbf{\textit{Link: \href{https://changminpark.github.io/rushmore}
{https://changminpark.github.io/rushmore}}}
\hfill

\vspace{0.5ex}
\Gap
\BulletItem
\textbf{Gesto} [EICS `19, PACM-HCI, Best Paper Honorable Mention] is a system that enables task 
automation for Android apps using gestures 
and voice commands. Using this system, a user can record a UI action sequence for an app, 
choose a gesture or a voice command to activate the UI action sequence, and later trigger 
the UI action sequence by the corresponding gesture/voice command.

\hspace{2ex}
\textbf{\textit{Link: \href{https://changminpark.github.io/gesto}
{https://changminpark.github.io/gesto}}}
\hfill

\vspace{0.5ex}
\Gap
\BulletItem
\textbf{Mimic} [ICSE `19] is an automated UI compatibility testing system for Android apps. 
Mimic is designed specifically for comparing the UI behavior of an app across different 
devices, different Android versions, and different app versions.

\hspace{2ex}
\textbf{\textit{Link: \href{https://changminpark.github.io/mimic}{https://changminpark.github.io/mimic}}}
\hfill

\vspace{0.5ex}
\Gap
\BulletItem
\textbf{Reptor} [MobiSys `17] enables open innovation in mobile platforms. Our technique allows third-party 
developers to modify, instrument, or extend platform API calls and deploy their modifications 
seamlessly. The uniqueness of our technique is that it enables modifications completely at 
the app layer without requiring any platform-level changes. 

\hspace{2ex}
\textbf{\textit{Link: \href{https://changminpark.github.io/reptor}{https://changminpark.github.io/reptor}}}
\hfill



%%%%%%%%%%%%%%%%%%%%%%%%%
%% RESEARCH EXPERIENCE %%
%%%%%%%%%%%%%%%%%%%%%%%%%
\Section
{Research Experience}
{Research Experience}
{PDF:ResearchExperience}

\Entry
\textbf{University at Buffalo}, The State University of New York

\Gap
\BulletItem
Ph.D. Research Assistant, RMS Lab
\hfill
\DatestampYD{Aug}{ `18} --
\DatestampY{Present}

\Gap
\BulletItem
Undergraduate Research Assistant, RMS Lab
\hfill
\DatestampYD{May}{ `16} --
\DatestampYD{Aug}{ `17}


%%%%%%%%%%%%%%%%%%%%%%%%%
%% TEACHING EXPERIENCE %%
%%%%%%%%%%%%%%%%%%%%%%%%%
\Section
{Teaching Experience}
{Teaching Experience}
{PDF:TeachingExperience}

\Entry
\textbf{University at Buffalo}, The State University of New York

\Gap
\BulletItem
CSE486/586: Distributed Systems
\hfill
\DatestampYD{Jan}{ `20} --
\DatestampY{May}{ `21}

\Gap
\BulletItem
CSE421/521: Operating Systems
\hfill
\DatestampYD{Aug}{ `17} --
\DatestampY{May}{ `18}



%%%%%%%%%%%%%%%%%%%%%
%% WORK EXPERIENCE %%
%%%%%%%%%%%%%%%%%%%%%
\Section
{Work Experience}
{Work Experience}
{PDF:WorkExperience}

\Entry
\textbf{Breeding Corporation}, South Korea

\Gap
\BulletItem
Founding Member \& CTO
\hfill
\DatestampYD{May}{ `20} --
\DatestampY{Jul}{ `21}

\Gap
\BulletItem
The first non-face-to-face platform service that connects dog owners and trainers
\hfill

\Gap
\BulletItem
Designed an app service and a business model - 
\textbf{\textit{Link: \href{https://changminpark.github.io/breeding}{https://changminpark.github.io/breeding}}}
\hfill




%%%%%%%%%%%%%%%%%%
%% PUBLICATIONS %%
%%%%%%%%%%%%%%%%%%
\Section
{Publications}
{Publications}
{PDF:Publications}


\SubSection
{Published}
{Published}
{PDF:Published}

% Declare a new group to limit the scope of \MaxNumberedItem to this subsection.
\begingroup
\renewcommand{\MaxNumberedItem}{[88]}

\Gap
\NumberedItem{[1]}
\underline{Chang Min Park}, Donghwi Kim, Deepesh Veersen Sidhwani, Andrew Fuchs, Arnob Paul,
Sung-Ju Lee, Karthik Dantu, and Stevn Y. Ko,
``Rushmore: Securely Displaying Static and Animated Images Using TrustZone'' in
\textit{Proceedings of the 19th Annual International Conference on Mobile Systems \textbf{(MobiSys)}
\DatestampYM{2021}.}


\Gap
\NumberedItem{[2]} 
\underline{Chang~Min~Park}, Taeyeon~Ki, Ali Ben Ali, 
Nikhil Sunil Pawar, Karthik Dantu, Steven Y. Ko, and Lukasz Ziarek, 
``Gesto: Mapping UI Events to Gestures and Voice'' in
\textit{Proceedings of 11th ACM SIGCHI Symposium on Engineering Interactive Computing Systems \textbf{(EICS)}
and \textit{Journal Proceedings of the ACM on Human-Computer Interaction \textbf{(PACM-HCI)}}, \DatestampYM{2019}{06}.}
\linebreak\textbf{Best Paper Honorable Mention}

\Gap
\NumberedItem{[3]}
Taeyeon Ki, \underline{Chang Min Park}, Karthik Dantu, Stevn Y. Ko, and Lukasz Ziarek, 
``Mimic: UI Compatibility Testing System for Android Apps'' in
\textit{Proceedings of the 41st International Conference on Software Engineering \textbf{(ICSE)}
, \DatestampYM{2019}{5}.}

\Gap
\NumberedItem{[4]}
\href{https://dl.acm.org/citation.cfm?id=3081341}
{Taeyeon Ki, Alexander Simeonov, Bhavika Pravin Jain, \underline{Chang Min Park},
Keshav Sharma, Karthik Dantu, Stevn Y. Ko, and Lukasz Ziarek, 
``Reptor: Enabling API Virtualization on Android for Platform Openness'' in 
\textit{Proceedings of the 15th Annual International Conference on Mobile Systems \textbf{(MobiSys)}
, \DatestampYM{2017}{6}}.}




%%%%%%%%%%%%%%%%%%%%%
%% POSTERS & DEMOS %%
%%%%%%%%%%%%%%%%%%%%%
\vspace{1ex}
\Section
{Posters and Demos}
{Posters and Demos}
{PDF:Posters and Demos}

\SubSection
{Posters}
{Posters}
{PDF:Posters}

\begingroup
\renewcommand{\MaxNumberedItem}{[88]}

\BigGap
\NumberedItem{[1]}
Harishankar Vishwanathan, \underline{Chang~Min~Park}, Sidharth Kumar Mishra, Karthik Dantu, 
Steven Y. Ko, and Lukasz Ziarek
``Poster: Partitioning Garbage Collection Between the Secure and Normal Worlds for Trusted Applications'' in
\textit{Proceedings of the 17th Annual International Conference on Mobile Systems \textbf{(MobiSys)}
\DatestampYM{2019}{6}}.

\Gap
\NumberedItem{[2]}
\underline{Chang~Min~Park}, Taeyeon~Ki, Ali Ben Ali, Karthik Dantu, Steven Y. Ko, 
and Lukasz Ziarek, 
``Enabling Dynamic Gesture Mapping with UI Events'' in
\textit{UB Graduate Research Conference and Alumni Symposium
\DatestampYM{2017}{09}.}


\BigGap
\SubSection
{Demos}
{Demos}
{PDF:Demos}

\begingroup
\renewcommand{\MaxNumberedItem}{[88]}

\BigGap
\NumberedItem{[1]}
\underline{Chang~Min~Park}, Taeyeon~Ki, Karthik Dantu, Steven Y. Ko, and Lukasz Ziarek, 
``Demo: Enabling Dynamic Gesture Mapping with UI Events'' in
\textit{Proceedings of the 15th Annual International Conference on Mobile Systems \textbf{(MobiSys)}
\DatestampYM{2017}{6}}.

\Gap
\NumberedItem{[2]}
Taeyeon Ki, Alexander Simeonov, \underline{Chang~Min~Park}, Karthik Dantu, Steven Y. Ko, 
and Lukasz Ziarek, 
``Demo: Reptor: Enabling API Virtualization on Android for Platform Openness'' in
\textit{Proceedings of the 15th Annual International Conference on Mobile Systems \textbf{(MobiSys)}
\DatestampYM{2017}{6}}.

\Gap
\NumberedItem{[3]}
Taeyeon~Ki, Alexander Simeonov, \underline{Chang~Min~Park}, Karthik Dantu, 
Steven Y. Ko, and Lukasz Ziarek, 
``Demo: Fully Automated UI Testing System for Large-scale Android Apps Using Multiple 
Devices'' in
\textit{Proceedings of the 15th Annual International Conference on Mobile Systems \textbf{(MobiSys)}
\DatestampYM{2017}{6}}.



%%%%%%%%%%%%%%%%%%%%%
%% AWARDS & GRANTS %%
%%%%%%%%%%%%%%%%%%%%%
\vspace{1ex}
\Section
{Awards \&\newline
Grants}
{Awards \& Grants}
{PDF:AwardsAndGrants}

\BulletItem
Excellence Award with \$80,000 Grant in K-Startup (ChungChung) Contest 
\hfill
\DatestampYM{2020}{11}
\begin{Detail}
\Item
\textit{Ministry of SMEs and Startups in South Korea}
\end{Detail}

\vspace{0.5ex}

\BulletItem
Top Award in Youth Startup Awards
\hfill
\DatestampYM{2020}{10}
\begin{Detail}
\Item
\textit{Youth and Future Corporation in South Korea}
\end{Detail}
\vspace{0.5ex}

\BulletItem
Pre-Startup Package with \$50,000 Grant 
\hfill
\DatestampYM{2020}{09}
\begin{Detail}
\Item
\textit{Ministry of SMEs and Startups in South Korea}
\end{Detail}
\vspace{0.5ex}

\BulletItem
Second Place in CSE PhD Poster Competition, University at Buffalo
\hfill
\DatestampYM{2019}{12}

\vspace{0.5ex}


\BulletItem
Best Paper Honorable Mention Award
\hfill
\DatestampYM{2019}{06}
\begin{Detail}
\Item
\textit{Proceedings of 11th ACM SIGCHI Symposium on Engineering Interactive Computing Systems \textbf{(EICS)}}
\end{Detail}
\vspace{0.5ex}

\Gap
\BulletItem
SEAS Dean’s Fellowship, University at Buffalo
\hfill
\DatestampY{2017}
%\begin{Detail}
%\Item
%For exceptional graduate students who have potential for an outstanding 
%graduate career 
%\end{Detail}

\vspace{0.5ex}

\Gap
\BulletItem
CSE Undergraduate Award for Research, University at Buffalo
\hfill
\DatestampYMD{2017}{05}{19}
%\begin{Detail}
%\Item
%Awarded to one graduating senior who has done exceptional research with a UB CSE faculty
%\end{Detail}

\vspace{0.5ex}

\Gap
\BulletItem
Dean's List, University at Buffalo
\hfill
\DatestampY{2012}


%%%%%%%%%%%%%%%%
%% ACTIVITIES %%
%%%%%%%%%%%%%%%%

%\Section
%{Activities}
%{Activities}
%{PDF:Activities}

%\Entry
%\textbf{Tau Beta Pi Engineering Honor Society},
%University at Buffalo

%\Gap
%\BulletItem
%Member
%\hfill
%\DatestampY{2016} --
%^\DatestampY{2017}


%%%%%%%%%%%%%%%%
%% References %%
%%%%%%%%%%%%%%%%
\vspace{5ex}
\Section
{References}
{References}
{PDF:References}

\Entry
\textbf{Steven Y. Ko} 

\hspace{1ex}Associate Professor, Computer Science and Engineering

\hspace{1ex}University at Buffalo, State University of New York and Simon Fraser University

\hspace{1ex}Email: steveyko@sfu.ca

\vspace{1ex}

\Gap
\Entry
\textbf{Karthik Dantu}

\hspace{1ex}Associate Professor, Computer Science and Engineering

\hspace{1ex}University at Buffalo, State University of New York

\hspace{1ex}Email: kdantu@buffalo.edu

\vspace{1ex}

\Gap
\Entry
\textbf{Lukasz Ziarek} 

\hspace{1ex}Associate Professor, Computer Science and Engineering

\hspace{1ex}University at Buffalo, State University of New York

\hspace{1ex}Email: lziarek@buffalo.edu

\vspace{1ex}

\Gap
\Entry
\textbf{Taeyeon Ki} 

\hspace{1ex}Senior Software Engineer, Samsung Research America

\hspace{1ex}Email: taeyeon.ki@samsung.com
\lastpage
\end{document}

