% !TEX TS-program = xelatex
% !TEX encoding = UTF-8 Unicode
% -*- coding: UTF-8; -*-
% vim: set fenc=utf-8

%%%%%%%%%%%%%%%%%%%%%%%%%%%%%%%%%%%%%%%%%%%%%%%%%%%%%%%%%%%%%%%%%
%% CV.tex
%% <https://github.com/zachscrivena/simple-resume-cv>
%% This is free and unencumbered software released into the
%% public domain; see <http://unlicense.org> for details.
%%%%%%%%%%%%%%%%%%%%%%%%%%%%%%%%%%%%%%%%%%%%%%%%%%%%%%%%%%%%%%%%%

% See "README.md" for instructions on compiling this document.

\documentclass[letterpaper,MMMyyyy,nonstopmode]{simpleresumecv}
% Class options:
% a4paper, letterpaper, nonstopmode, draftmode
% MMMyyyy, ddMMMyyyy, MMMMyyyy, ddMMMMyyyy, yyyyMMdd, yyyyMM, yyyy

%%%%%%%%%%%%%%%%%%%%%%%%%%%%%%%%%%%%%%%%%%%%%%%%%%%%%%%%%%%%%%%%%
%% PREAMBLE.
%%%%%%%%%%%%%%%%%%%%%%%%%%%%%%%%%%%%%%%%%%%%%%%%%%%%%%%%%%%%%%%%%

% CV Info (to be customized).
\newcommand{\CVAuthor}{Chang Min Park}
\newcommand{\CVTitle}{Chang Min Park's CV}
\newcommand{\CVNote}{CV compiled on {\today}}
\newcommand{\CVWebpage}{http://www.beyondthegeek.com/}

% PDF settings and properties.
\hypersetup{
pdftitle={\CVTitle},
pdfauthor={\CVAuthor},
pdfsubject={\CVWebpage},
pdfcreator={XeLaTeX},
pdfproducer={},
pdfkeywords={},
unicode=true,
bookmarks=true,
bookmarksopen=true,
pdfstartview=FitH,
pdfpagelayout=OneColumn,
pdfpagemode=UseOutlines,
hidelinks,
breaklinks}

% Shorthand.
\newcommand{\Code}[1]{\mbox{\textbf{#1}}}
\newcommand{\CodeCommand}[1]{\mbox{\textbf{\textbackslash{#1}}}}

%%%%%%%%%%%%%%%%%%%%%%%%%%%%%%%%%%%%%%%%%%%%%%%%%%%%%%%%%%%%%%%%%
%% ACTUAL DOCUMENT.
%%%%%%%%%%%%%%%%%%%%%%%%%%%%%%%%%%%%%%%%%%%%%%%%%%%%%%%%%%%%%%%%%

\begin{document}

%%%%%%%%%%%%%%%
% TITLE BLOCK %
%%%%%%%%%%%%%%%

\Title{\CVAuthor}

\begin{SubTitle}
University at Buffalo, The State University of New York
\par
\href{mailto:cpark22@buffalo.edu}
{cpark22@buffalo.edu}
\,\SubBulletSymbol\,
+1\,(716)\,598-7331
\,\SubBulletSymbol\,
\href{\CVWebpage}
{\url{\CVWebpage}}
\end{SubTitle}

\begin{Body}

%%%%%%%%%%%%%%%
%% Interests %%
%%%%%%%%%%%%%%%

\Section
{Interests}
{Interests}
{PDF:Interests}

\Entry
Mobile Systems, UI Automation, Platform API Virtualization, and Automated testing system.

%%%%%%%%%%%%%%%%%%%%%%
%% TECHNICAL SKILLS %%
%%%%%%%%%%%%%%%%%%%%%%

\Section
{Technical Skills}
{Technical Skills}
{PDF:Technical Skills}

\Entry
Android(System, Platform API, and App Devleopment), Soot API, Firebase Realtime Database, Shell, Java, Python, C++, and Linux OS
\BigGap

%%%%%%%%%%%%%%%
%% EDUCATION %%
%%%%%%%%%%%%%%%

\Section
{Education}
{Education}
{PDF:Education}

\Entry
\textbf{University at Buffalo}, The State University of New York

\Gap
\BulletItem
Ph.D. in Computer Science and Engineering
\hfill
\DatestampYD{Aug}{ `17} --
\DatestampY{Present}
\begin{Detail}
\SubBulletItem
Advisor:
Prof.~Steven~Y.~Ko
\SubBulletItem
Focus:
Mobile computing and applications.
\end{Detail}

\Gap
\BulletItem
B.S. in Computer Science
\hfill
\DatestampYD{Aug}{ `11} --
\DatestampYD{May}{ `17}
\begin{Detail}
\SubBulletItem
Magna Cum Laude
\SubBulletItem
Jun `13 -- Mar `15: Served Military Service in Republic of Korea
\end{Detail}

\BigGap
\Entry
\textbf{Yonsei University}, Republic of Korea

\Gap
\BulletItem
SUNY Study Abroad Program
\hfill
\DatestampYD{Summer}{ `12}

\BigGap
\Entry
\textbf{Relevant Courses:} Advanced Computer System, Advanced Programming Language, Operating System, Realtime Embed System,
Data Structure, Modern Network Concepts, Data Intensive Computing,  and Computer Security.

%%%%%%%%%%%%%%%%%%%%%%%
%% RESEARCH OVERVIEW %%
%%%%%%%%%%%%%%%%%%%%%%%

\Section
{Research Overview}
{Research Overview}
{PDF:ResearchOverview}

\Entry
Over the course from my undergraduate to Ph.D., I have focused on mobile systems and automation.

\Gap
\BulletItem
\textbf{Gesto} is a system that enables task automation for Android apps using gestures 
and voice commands. Using this system, a user can record a UI action sequence for an app, 
choose a gesture or a voice command to activate the UI action sequence, and later trigger 
the UI action sequence by the corresponding gesture/voice command.
\hfill

\Gap
\BulletItem
\textbf{Reptor} enables open innovation in mobile platforms. Our technique allows third-party 
developers to modify, instrument, or extend platform API calls and deploy their modifications 
seamlessly. The uniqueness of our technique is that it enables modifications completely at 
the app layer without requiring any platform-level changes. 
\hfill

\Gap
\BulletItem
\textbf{Mimic} is an automated UI compatibility testing system for Android apps. 
Mimic is designed specifically for comparing the UI behavior of an app across different 
devices, different Android versions, and different app versions.
\hfill


%%%%%%%%%%%%%%%%%%%%%%%%%
%% RESEARCH EXPERIENCE %%
%%%%%%%%%%%%%%%%%%%%%%%%%

\Section
{Research Experience}
{Research Experience}
{PDF:ResearchExperience}

\Entry
\textbf{University at Buffalo}, The State University of New York

\Gap
\BulletItem
Ph.D. Research Assistant, RMS Lab
\hfill
\DatestampYD{Aug}{ `18} --
\DatestampY{Present}
\begin{Detail}
\SubBulletItem
Project:
Mapping UI Events to Gestures and Voice, and Automated Testing System.
\end{Detail}

\Gap
\BulletItem
Undergraduate Research Assistant, RMS Lab
\hfill
\DatestampYD{May}{ `16} --
\DatestampYD{Aug}{ `17}
\begin{Detail}
\SubBulletItem
Project:
Android Platform API Virtualization
\end{Detail}

%%%%%%%%%%%%%%%%%%%%%%%%%
%% TEACHING EXPERIENCE %%
%%%%%%%%%%%%%%%%%%%%%%%%%

\Section
{Teaching Experience}
{Teaching Experience}
{PDF:TeachingExperience}

\Entry
\textbf{University at Buffalo}, The State University of New York

\Gap
\BulletItem
CSE421/521: Operating Systems
\hfill
\DatestampYD{Aug}{ `17} --
\DatestampY{May}{ `18}
\begin{Detail}
\SubBulletItem
Design and Implementation of Operating Systems
\SubBulletItem
Project: Pintos Programming
\end{Detail}



%%%%%%%%%%%%%%%%%%
%% PUBLICATIONS %%
%%%%%%%%%%%%%%%%%%

\Section
{Publications}
{Publications}
{PDF:Publications}

\SubSection
{Conferences}
{Conferences}
{PDF:Conferences}

% Declare a new group to limit the scope of \MaxNumberedItem to this subsection.
\begingroup
\renewcommand{\MaxNumberedItem}{[88]}

\BigGap
\NumberedItem{[1]}
\underline{Chang~Min~Park}, Taeyeon~Ki, Ali Ben Ali, 
Nikhil Sunil Pawar, Karthik Dantu, Steven Y. Ko, and Lukasz Ziarek, 
``Gesto: Mapping UI Events to Gestures and Voice''
forthcoming in
\textit{Proceedings of 11th ACM SIGCHI Symposium on Engineering Interactive Computing Systems (EICS)
\DatestampYM{2019}{06}.}

\Gap
\NumberedItem{[2]}
\href{https://dl.acm.org/citation.cfm?id=3081341}
{Taeyeon Ki, Alexander Simeonov, Bhavika Pravin Jain, \underline{Chang Min Park},
Keshav Sharma, Karthik Dantu, Stevn Y. Ko, and Lukasz Ziarek, 
``Reptor: Enabling API Virtualization on Android for Platform Openness'' in 
\textit{Proceedings of the 15th Annual International Conference on Mobile Systems (MobiSys)
\DatestampYM{2017}{6}}.}

\Gap
\NumberedItem{[3]}
Taeyeon Ki, \underline{Chang Min Park}, Karthik Dantu, Stevn Y. Ko, and Lukasz Ziarek, 
``Mimic: UI Compatibility Testing System for Android Apps'' submitted in 
\textit{Proceedings of the 41th International Conference on Software Engineering (ICSE)
\DatestampYM{2019}.}


\BigGap
\SubSection
{Journals}
{Journals}
{PDF:Journals}

% Declare a new group to limit the scope of \MaxNumberedItem to this subsection.
\begingroup
\renewcommand{\MaxNumberedItem}{[88]}

\BigGap
\NumberedItem{[1]}
\underline{Chang~Min~Park}, Taeyeon~Ki, Ali Ben Ali, 
Nikhil Sunil Pawar, Karthik Dantu, Steven Y. Ko, and Lukasz Ziarek, 
``Gesto: Mapping UI Events to Gestures and Voice''
forthcoming in \textit{Journal Proceedings of the ACM on Human-Computer Interaction - EICS
\DatestampY{2019}.}

\Gap
\NumberedItem{[2]}
Taeyeon Ki, Alexander Simeonov, \underline{Chang Min Park},
Karthik Dantu, Stevn Y. Ko, and Lukasz Ziarek, 
``Reptor: Enabling API Virtualization on Android for Platform Openness'' submitted in 
\textit{ACM Transactions on Software Engineering and Methodology (TOSEM)
\DatestampYM{2018}}}.}

%%%%%%%%%%%%%%%%%%%%%
%% POSTERS & DEMOS %%
%%%%%%%%%%%%%%%%%%%%%

\Section
{Posters and Demos}
{Posters and Demos}
{PDF:Posters and Demos}

\SubSection
{Posters}
{Posters}
{PDF:Posters}

\begingroup
\renewcommand{\MaxNumberedItem}{[88]}

\BigGap
\NumberedItem{[1]}
\underline{Chang~Min~Park}, Taeyeon~Ki, Ali Ben Ali, Karthik Dantu, Steven Y. Ko, 
and Lukasz Ziarek, 
``Enabling Dynamic Gesture Mapping with UI Events'' in
\textit{UB Graduate Research Conference and Alumni Symposium
\DatestampYM{2017}{09}.}

\BigGap
\SubSection
{Demos}
{Demos}
{PDF:Demos}

\begingroup
\renewcommand{\MaxNumberedItem}{[88]}

\BigGap
\NumberedItem{[1]}
\underline{Chang~Min~Park}, Taeyeon~Ki, Karthik Dantu, Steven Y. Ko, and Lukasz Ziarek, 
``Demo: Enabling Dynamic Gesture Mapping with UI Events'' in
\textit{Proceedings of the 15th Annual International Conference on Mobile Systems (MobiSys)
\DatestampYM{2017}{6}}.}

\Gap
\NumberedItem{[2]}
Taeyeon Ki, Alexander Simeonov, \underline{Chang~Min~Park}, Karthik Dantu, Steven Y. Ko, 
and Lukasz Ziarek, 
``Demo: Reptor: Enabling API Virtualization on Android for Platform Openness'' in
\textit{Proceedings of the 15th Annual International Conference on Mobile Systems (MobiSys)
\DatestampYM{2017}{6}}.}

\Gap
\NumberedItem{[3]}
Taeyeon~Ki, Alexander Simeonov, \underline{Chang~Min~Park}, Karthik Dantu, 
Steven Y. Ko, and Lukasz Ziarek, 
``Demo: Fully Automated UI Testing System for Large-scale Android Apps Using Multiple 
Devices'' in
\textit{Proceedings of the 15th Annual International Conference on Mobile Systems (MobiSys)
\DatestampYM{2017}{6}}.}



%%%%%%%%%%%%%%%%%%%%%
%% HONORS & AWARDS %%
%%%%%%%%%%%%%%%%%%%%%

\Section
{Honors \&\newline
Awards}
{Honors \& Awards}
{PDF:HonorsAndAwards}


\BulletItem
Dean’s Fellowship Award along with a Full Financial Aid
\hfill
\DatestampY{2017} --
\DatestampY{2018}
\begin{Detail}
\Item
For exceptional graduate students who have potential for an outstanding 
graduate career. 
\end{Detail}

\Gap
\BulletItem
CSE Undergraduate Award for Research, University at Buffalo
\hfill
\DatestampYMD{2017}{05}{19}

\Gap
\BulletItem
Dean's List, University at Buffalo
\hfill
\DatestampY{2012}



%%%%%%%%%%%%%%%%
%% ACTIVITIES %%
%%%%%%%%%%%%%%%%

\Section
{Activities}
{Activities}
{PDF:Activities}

\Entry
\textbf{Tau Beta Pi Engineering Honor Society},
University at Buffalo

\Gap
\BulletItem
Member
\hfill
\DatestampY{2016} --
\DatestampY{2017}



\end{document}
