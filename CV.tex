% !TEX TS-program = xelatex
% !TEX encoding = UTF-8 Unicode
% -*- coding: UTF-8; -*-
% vim: set fenc=utf-8

%%%%%%%%%%%%%%%%%%%%%%%%%%%%%%%%%%%%%%%%%%%%%%%%%%%%%%%%%%%%%%%%%
%% CV.tex
%% <https://github.com/zachscrivena/simple-resume-cv>
%% This is free and unencumbered software released into the
%% public domain; see <http://unlicense.org> for details.
%%%%%%%%%%%%%%%%%%%%%%%%%%%%%%%%%%%%%%%%%%%%%%%%%%%%%%%%%%%%%%%%%

% See "README.md" for instructions on compiling this document.

\documentclass[letterpaper,MMMyyyy,nonstopmode]{simpleresumecv}
% Class options:
% a4paper, letterpaper, nonstopmode, draftmode
% MMMyyyy, ddMMMyyyy, MMMMyyyy, ddMMMMyyyy, yyyyMMdd, yyyyMM, yyyy

%%%%%%%%%%%%%%%%%%%%%%%%%%%%%%%%%%%%%%%%%%%%%%%%%%%%%%%%%%%%%%%%%
%% PREAMBLE.
%%%%%%%%%%%%%%%%%%%%%%%%%%%%%%%%%%%%%%%%%%%%%%%%%%%%%%%%%%%%%%%%%

% CV Info (to be customized).
\newcommand{\CVAuthor}{Chang Min Park}
\newcommand{\CVTitle}{Chang Min Park's CV}
\newcommand{\CVNote}{CV compiled on {\today}}
\newcommand{\CVWebpage}{http://www.beyondthegeek.com/}

% PDF settings and properties.
\hypersetup{
pdftitle={\CVTitle},
pdfauthor={\CVAuthor},
pdfsubject={\CVWebpage},
pdfcreator={XeLaTeX},
pdfproducer={},
pdfkeywords={},
unicode=true,
bookmarks=true,
bookmarksopen=true,
pdfstartview=FitH,
pdfpagelayout=OneColumn,
pdfpagemode=UseOutlines,
hidelinks,
breaklinks}

% Shorthand.
\newcommand{\Code}[1]{\mbox{\textbf{#1}}}
\newcommand{\CodeCommand}[1]{\mbox{\textbf{\textbackslash{#1}}}}

%%%%%%%%%%%%%%%%%%%%%%%%%%%%%%%%%%%%%%%%%%%%%%%%%%%%%%%%%%%%%%%%%
%% ACTUAL DOCUMENT.
%%%%%%%%%%%%%%%%%%%%%%%%%%%%%%%%%%%%%%%%%%%%%%%%%%%%%%%%%%%%%%%%%

\begin{document}

%%%%%%%%%%%%%%%
% TITLE BLOCK %
%%%%%%%%%%%%%%%

\Title{\CVAuthor}

\begin{SubTitle}
University at Buffalo, The State University of New York
\par
\href{mailto:cpark22@buffalo.edu}
{cpark22@buffalo.edu}
\,\SubBulletSymbol\,
+1\,(716)\,598-7331
\,\SubBulletSymbol\,
\href{\CVWebpage}
{\url{\CVWebpage}}
\end{SubTitle}

\begin{Body}

%%%%%%%%%%%%%%%
%% Interests %%
%%%%%%%%%%%%%%%

\Section
{Interests}
{Interests}
{PDF:Interests}

\Entry
Systems Challenges in Mobile Systems, Automated Software Analysis, and UI Testing.

%%%%%%%%%%%%%%%%%%%%%%
%% TECHNICAL SKILLS %%
%%%%%%%%%%%%%%%%%%%%%%

\Section
{Technical Skills}
{Technical Skills}
{PDF:Technical Skills}

\Entry
Android Internals and App Devleopment, Bytecode Instrumentation Tools (Soot), Firebase
Realtime Database, Java, Python, C++, and Linux OS
\BigGap

%%%%%%%%%%%%%%%
%% EDUCATION %%
%%%%%%%%%%%%%%%

\Section
{Education}
{Education}
{PDF:Education}

\Entry
\textbf{University at Buffalo}, The State University of New York

\Gap
\BulletItem
Ph.D. in Computer Science and Engineering
\hfill
\DatestampYD{Aug}{ `17} --
\DatestampY{Present}
\begin{Detail}
\SubBulletItem
Advisor:
Prof.~Steven~Y.~Ko
\SubBulletItem
Focus:
Systems Challenges in Mobile Computing
\end{Detail}

\Gap
\BulletItem
B.S. in Computer Science
\hfill
\DatestampYD{Aug}{ `11} --
\DatestampYD{May}{ `17}
\begin{Detail}
\SubBulletItem
Magna Cum Laude
\SubBulletItem
Jun `13 -- Mar `15: Served Military Service in Republic of Korea
\end{Detail}

\BigGap
\Entry
\textbf{Yonsei University}, Republic of Korea

\Gap
\BulletItem
SUNY Study Abroad Program
\hfill
\DatestampYD{Summer}{ `12}

\BigGap
\Entry
\textbf{Relevant Courses:} Advanced Computer Systems, Advanced Programming Languages, Operating Systems, 
Realtime Embedded Systems, Modern Network Concepts, Data Intensive Computing,  and Computer Security.

%%%%%%%%%%%%%%%%%%%%%%%
%% RESEARCH OVERVIEW %%
%%%%%%%%%%%%%%%%%%%%%%%

\Section
{Research Overview}
{Research Overview}
{PDF:ResearchOverview}

%\Entry
%Currently, I'm working on two projects below.
%\Gap
%\BulletItem
%\textbf{Immix} garbage collection is powerful and well-known, but Dart VM, that Google's 
%new framework Flutter uses, has generation garbage collection. My goal is implementing the Immix garbage 
%collection in Dart VM and analyzing its performance.
%\hfill

%\Gap
%\BulletItem
%\textbf{AR Authentication on Mobile} can be solved by using TrustZone feature that has two separate environments, 
%normal and secure. Depending on user's authentication level, a mobile device should be able to authenticate 
%augmented objects. I'm currently working on implementing authentication of AR using TrustZone enabled board. 
%\hfill

\Entry
%Over the course from my undergraduate to Ph.D., I have focused on mobile systems.
Through my undergraduate and graduate studies, my research has focused on mobile systems.

\Gap
\BulletItem
\textbf{Gesto} [EICS `19, PACM-HCI] is a system that enables task automation for Android apps using gestures 
and voice commands. Using this system, a user can record a UI action sequence for an app, 
choose a gesture or a voice command to activate the UI action sequence, and later trigger 
the UI action sequence by the corresponding gesture/voice command.

\hspace{2ex}
\textbf{\textit{Link: \href{http://beyondthegeek.com/portfolio/gesto-eics-19/}
{http://beyondthegeek.com/portfolio/gesto-eics-19/}}}
\hfill

\Gap
\BulletItem
\textbf{Reptor} [MobiSys `17] enables open innovation in mobile platforms. Our technique allows third-party 
developers to modify, instrument, or extend platform API calls and deploy their modifications 
seamlessly. The uniqueness of our technique is that it enables modifications completely at 
the app layer without requiring any platform-level changes. 

\hspace{2ex}
\textbf{\textit{Link: \href{http://reptor.cse.buffalo.edu/}{http://reptor.cse.buffalo.edu/}}}
\hfill

\Gap
\BulletItem
\textbf{Mimic} [ICSE `19] is an automated UI compatibility testing system for Android apps. 
Mimic is designed specifically for comparing the UI behavior of an app across different 
devices, different Android versions, and different app versions.
\hfill


%%%%%%%%%%%%%%%%%%%%%%%%%
%% RESEARCH EXPERIENCE %%
%%%%%%%%%%%%%%%%%%%%%%%%%

\Section
{Research Experience}
{Research Experience}
{PDF:ResearchExperience}

\Entry
\textbf{University at Buffalo}, The State University of New York

\Gap
\BulletItem
Ph.D. Research Assistant, RMS Lab
\hfill
\DatestampYD{Aug}{ `18} --
\DatestampY{Present}
\begin{Detail}
\SubBulletItem
Project:
Mapping UI Events to Gestures and Voice, and Automated Testing of Mobile Devices.
\end{Detail}

\Gap
\BulletItem
Undergraduate Research Assistant, RMS Lab
\hfill
\DatestampYD{May}{ `16} --
\DatestampYD{Aug}{ `17}
\begin{Detail}
\SubBulletItem
Project:
Android Platform API Virtualization
\end{Detail}

%%%%%%%%%%%%%%%%%%%%%%%%%
%% TEACHING EXPERIENCE %%
%%%%%%%%%%%%%%%%%%%%%%%%%

\Section
{Teaching Experience}
{Teaching Experience}
{PDF:TeachingExperience}

\Entry
\textbf{University at Buffalo}, The State University of New York

\Gap
\BulletItem
CSE421/521: Operating Systems
\hfill
\DatestampYD{Aug}{ `17} --
\DatestampY{May}{ `18}
\begin{Detail}
\SubBulletItem
Design and Implementation of Operating Systems
\SubBulletItem
Project: Pintos Programming
\end{Detail}



%%%%%%%%%%%%%%%%%%
%% PUBLICATIONS %%
%%%%%%%%%%%%%%%%%%

\Section
{Publications}
{Publications}
{PDF:Publications}

\SubSection
{Published}
{Published}
{PDF:Published}

% Declare a new group to limit the scope of \MaxNumberedItem to this subsection.
\begingroup
\renewcommand{\MaxNumberedItem}{[88]}

\BigGap
\NumberedItem{[1]} 
\underline{Chang~Min~Park}, Taeyeon~Ki, Ali Ben Ali, 
Nikhil Sunil Pawar, Karthik Dantu, Steven Y. Ko, and Lukasz Ziarek, 
``Gesto: Mapping UI Events to Gestures and Voice''
forthcoming in
\textit{Proceedings of 11th ACM SIGCHI Symposium on Engineering Interactive Computing Systems \textbf{(EICS)}
and \textit{Journal Proceedings of the ACM on Human-Computer Interaction \textbf{(PACM-HCI)}}
, \DatestampYM{2019}{06}.}

\Gap
\NumberedItem{[2]}
Taeyeon Ki, \underline{Chang Min Park}, Karthik Dantu, Stevn Y. Ko, and Lukasz Ziarek, 
``Mimic: UI Compatibility Testing System for Android Apps'' forthcoming in
\textit{Proceedings of the 41st International Conference on Software Engineering \textbf{(ICSE)}
, \DatestampYM{2019}{5}.}

\Gap
\NumberedItem{[3]}
\href{https://dl.acm.org/citation.cfm?id=3081341}
{Taeyeon Ki, Alexander Simeonov, Bhavika Pravin Jain, \underline{Chang Min Park},
Keshav Sharma, Karthik Dantu, Stevn Y. Ko, and Lukasz Ziarek, 
``Reptor: Enabling API Virtualization on Android for Platform Openness'' in 
\textit{Proceedings of the 15th Annual International Conference on Mobile Systems \textbf{(MobiSys)}
, \DatestampYM{2017}{6}}.}


%\SubSection
%{Submitted}
%{Submitted}
%{PDF:Submitted}

% Declare a new group to limit the scope of \MaxNumberedItem to this subsection.
%\begingroup
%\renewcommand{\MaxNumberedItem}{[88]}


%%%%%%%%%%%%%%%%%%%%%
%% POSTERS & DEMOS %%
%%%%%%%%%%%%%%%%%%%%%

\Section
{Posters and Demos}
{Posters and Demos}
{PDF:Posters and Demos}

\SubSection
{Posters}
{Posters}
{PDF:Posters}

\begingroup
\renewcommand{\MaxNumberedItem}{[88]}

\BigGap
\NumberedItem{[1]}
\underline{Chang~Min~Park}, Taeyeon~Ki, Ali Ben Ali, Karthik Dantu, Steven Y. Ko, 
and Lukasz Ziarek, 
``Enabling Dynamic Gesture Mapping with UI Events'' in
\textit{UB Graduate Research Conference and Alumni Symposium
\DatestampYM{2017}{09}.}

\BigGap
\SubSection
{Demos}
{Demos}
{PDF:Demos}

\begingroup
\renewcommand{\MaxNumberedItem}{[88]}

\BigGap
\NumberedItem{[1]}
\underline{Chang~Min~Park}, Taeyeon~Ki, Karthik Dantu, Steven Y. Ko, and Lukasz Ziarek, 
``Demo: Enabling Dynamic Gesture Mapping with UI Events'' in
\textit{Proceedings of the 15th Annual International Conference on Mobile Systems \textbf{(MobiSys)}
\DatestampYM{2017}{6}}.

\Gap
\NumberedItem{[2]}
Taeyeon Ki, Alexander Simeonov, \underline{Chang~Min~Park}, Karthik Dantu, Steven Y. Ko, 
and Lukasz Ziarek, 
``Demo: Reptor: Enabling API Virtualization on Android for Platform Openness'' in
\textit{Proceedings of the 15th Annual International Conference on Mobile Systems \textbf{(MobiSys)}
\DatestampYM{2017}{6}}.

\Gap
\NumberedItem{[3]}
Taeyeon~Ki, Alexander Simeonov, \underline{Chang~Min~Park}, Karthik Dantu, 
Steven Y. Ko, and Lukasz Ziarek, 
``Demo: Fully Automated UI Testing System for Large-scale Android Apps Using Multiple 
Devices'' in
\textit{Proceedings of the 15th Annual International Conference on Mobile Systems \textbf{(MobiSys)}
\DatestampYM{2017}{6}}.



%%%%%%%%%%%%%%%%%%%%%
%% HONORS & AWARDS %%
%%%%%%%%%%%%%%%%%%%%%

\Section
{Honors \&\newline
Awards}
{Honors \& Awards}
{PDF:HonorsAndAwards}


\BulletItem
UB SEAS Dean’s Fellowship
\hfill
\DatestampY{2017} --
\DatestampY{2018}
\begin{Detail}
\Item
For exceptional graduate students who have potential for an outstanding 
graduate career. 
\end{Detail}

\Gap
\BulletItem
CSE Undergraduate Award for Research, University at Buffalo
\hfill
\DatestampYMD{2017}{05}{19}
\begin{Detail}
\Item
Awarded to one graduating senior who has done exceptional research with a UB CSE faculty. 
\end{Detail}

\Gap
\BulletItem
Dean's List, University at Buffalo
\hfill
\DatestampY{2012}


%%%%%%%%%%%%%%%%
%% ACTIVITIES %%
%%%%%%%%%%%%%%%%

\Section
{Activities}
{Activities}
{PDF:Activities}

\Entry
\textbf{Tau Beta Pi Engineering Honor Society},
University at Buffalo

\Gap
\BulletItem
Member
\hfill
\DatestampY{2016} --
\DatestampY{2017}


%%%%%%%%%%%%%%%%
%% References %%
%%%%%%%%%%%%%%%%

\Section
{References}
{References}
{PDF:References}

\Entry
\textbf{Steven Y. Ko} 

\hspace{1ex}Associate Professor, Computer Science and Engineering

\hspace{1ex}University at Buffalo, State University of New York

\hspace{1ex}Email: stevko@buffalo.edu

\Gap
\Entry
\textbf{Karthik Dantu}

\hspace{1ex}Assistant Professor, Computer Science and Engineering

\hspace{1ex}University at Buffalo, State University of New York

\hspace{1ex}Email: kdantu@buffalo.edu

\Gap
\Entry
\textbf{Lukasz Ziarek} 

\hspace{1ex}Assistant Professor, Computer Science and Engineering

\hspace{1ex}University at Buffalo, State University of New York

\hspace{1ex}Email: lziarek@buffalo.edu

\Gap
\Entry
\textbf{Taeyeon Ki} 

\hspace{1ex}Senior Software Engineer

\hspace{1ex}Samsung Research America

\hspace{1ex}Email: taeyeon.ki@samsung.com

\end{document}
